\documentclass[]{resources/unswthesis}

%%% Class options:

%  undergrad (default)
%  hdr

%  11pt (default)
%  12pt

%  final (default)
%  draft

%  oneside (default for hdr)
%  twoside (default for undergrad)


%% Thesis details
\thesistitle{Requirements to theses submitted in the Faculty of Engineering}
\thesisschool{School of Computer Science and Engineering}
\thesisauthor{Andrew Jin-Meng Wong}
\thesisZid{z5206677}
\thesistopic{"Smart" Vacuum Cleaners - An Audit Into The Security and Integrity of IoT Systems}
\thesisdegree{Bachelor of Engineering in Computer Engineering}
\thesisdate{November 2021}
\thesissupervisor{Prof.\ Richard Buckland}% for undergrad theses only


%% My own LaTeX macros, definitions, etc
%%%% Shortcuts
\newcommand{\num}[2]{\mbox{#1\,#2}}			% num with units

%%%% Symbols
\newcommand{\yes}{\ensuremath{\surd}\xspace}		% Tick mark
\newcommand{\no}{\ensuremath{\times}\xspace}		% Cross mark
\newcommand{\by}{\ensuremath{\times}\xspace}		% XXX x XXX
\newcommand{\bAND}{\ensuremath{\wedge}\xspace}		% Bool. /\
\newcommand{\bOR}{\ensuremath{\vee}\xspace}		% Bool. \/
\newcommand{\becomes}{\ensuremath{\rightarrow}\xspace}	% -->

%%%% Custom environments

% Centered tabular with single spacing
\newenvironment{ctabular}[1]
    {\par\begin{sspacing}\begin{center}\begin{tabular}{#1}}%
    {\end{tabular}\end{center}\end{sspacing}}


%%%% Our default level for display in TOC - subsubsections
\setcounter{tocdepth}{2}



\begin{document}

%% pages in the ``frontmatter'' section have roman numeral page number
\frontmatter  
\maketitle

\chapter*{Abstract}\label{abstract}
This document describes the requirements to theses submitted for the
\ThesisDegreeName\ degree at the \ThesisSchoolName.  Requirements
described are that of both of context and layout of the theses.  The
document is written using the \LaTeX\ template provided by the school.

\chapter*{Acknowledgements}\label{ack}

This work has been inspired by the labours of numerous academics in
the Faculty of Engineering at UNSW who have endeavoured, over the years, to
encourage students to present beautiful concepts using beautiful
typography.

Further inspiration has come from Donald Knuth who designed \TeX, for
typesetting technical (and non-technical) material with elegance and
clarity; and from Leslie Lamport who contributed \LaTeX, which makes
\TeX\ usable by mortal engineers.

John Zaitseff, an honours student in CSE at the time, created the
first version of the UNSW Thesis \LaTeX\ class and the author of the
current version is indebted to his work.

\chapter*{Abbreviations}\label{abbr}
\begin{description}
\item[BE] Bachelor of Engineering
\item[\LaTeX] A document preparation computer program
\item[PhD] Doctor of Philosophy
\end{description}


\tableofcontents
\listoffigures  % if required
\listoftables  % if required

%% pages in the ``mainmatter'' section have arabic page numbers and chapters are numbered
\mainmatter

\hypertarget{introduction}{%
\chapter{Introduction}\label{introduction}}

Consumer grade Internet of Things (IoT) devices have become widely
adopted with continuously growing demand. With demands for such devices
growing by 12\% each year (Research \& Markets 2021), this AU\$130B
industry has cordially invited thousands of households to invest in
smart devices such as light bulbs, fans, televisions and fridges. Giving
the abundance and affordability of these products, IoT devices have
become an integral part of many homes, where 4 in 5 consumers would be
more inclined to choose a property over another if the former were to
have such technologies (Brown 2015).

Although convenient, these devices come with hidden costs and risks.
Behind the seemingly `simple', `smart' and `secure' product features
that attract the general populus lies a hidden complex network of
services and devices, where functionality is often obscured and private.
Without the transparency of what data is being sent, and of where that
data is being sent to, consumers inevitably pay for convenience with not
only their money but with their privacy and security (Mehic et al.
2019).

Whilst manufacturers and vendors claim to be secure and/or confidential
in how they treat \textbf{UGC} and \textbf{PII}, it lies evident from
various incidents that we cannot completely trust such claims. From
involuntary exposure of leaked Facebook user data (Abrams 2021), to
rumours of corporations monetising user data without consent (Jones
2017), there lies an equal need for consumers to understand the terms of
service to which they agree to, but additionally for companies to be
audited against those very same terms of service.

The infrastructural security and product security of IoT devices must
also be scrutinised, given the rapid product lifecycle of IoT
developments (Giese 2021). As security is often not a sellable feature
in contrast to new products and most mistakenly, convenience, proper and
wholistic security precautions are often overlooked by companies who are
more concerned with profits and high return on investments.
Consequently, the prevalence of malicious actors in the cyberworld is
alarming, where the overall lack of security awareness between consumers
invites target devices to be easily accessed with default passwords or
through unpatched vulnerabilities\footnote{https://www.shodan.io/search?query=webcam}.

The black-box nature of IoT network communication raises both privacy
and security concerns that may often be overlooked or trivialised in
exchange for convenience. This this will involve the audit of an
internet-connected robotic vacuum cleaner (Roborock S6) to assess the
internal operations and nature of data that is transmitted, as to
investigate any potential vulnerabilities that may render the device
insecure.

The outline of this thesis is as follows: In Chapter 2, we study the
motivations behind wanting to audit the privacy and security of IoT
systems. In Chapter 3, we will review existing research, results and
methods that comprise the current state of the art of security research
on robot vacuum cleaners. Gaps in existing research will then be
formalised through the thesis plan in Chapter 4. Finally we will
conclude this report in a discussion of preliminary results that will be
carried forwards into the later stages of research

\hypertarget{refs}{}
\begin{CSLReferences}{1}{0}
\leavevmode\vadjust pre{\hypertarget{ref-LawrenceAbrams-2021}{}}%
Abrams, L., 2021. 533 million facebook users' phone numbers leaked on
hacker forum. Available at:
\url{https://www.bleepingcomputer.com/news/security/533-million-facebook-users-phone-numbers-leaked-on-hacker-forum/}.

\leavevmode\vadjust pre{\hypertarget{ref-RichBrown-2015}{}}%
Brown, R., 2015. Smart homes can pay off when it's time to sell.
Available at:
\url{https://www.cnet.com/home/smart-home/what-happens-when-you-sell-your-smart-house/}.

\leavevmode\vadjust pre{\hypertarget{ref-DennisGiese-2021}{}}%
Giese, D., 2021. Smart home security \& privacy.

\leavevmode\vadjust pre{\hypertarget{ref-RhettJones-2017}{}}%
Jones, R., 2017. Roomba's next big step is selling maps of your home to
the highest bidder. Available at:
\url{https://gizmodo.com/roombas-next-big-step-is-selling-maps-of-your-home-to-t-1797187829}.

\leavevmode\vadjust pre{\hypertarget{ref-8939043}{}}%
Mehic, M., Selimovic, N. \& Komosny, D., 2019.
\href{https://doi.org/10.1109/ICAT47117.2019.8939043}{About the
connectivity of xiaomi internet-of-things smart home devices}. In
\emph{2019 XXVII international conference on information, communication
and automation technologies (ICAT)}. pp. 1--6.

\leavevmode\vadjust pre{\hypertarget{ref-ResearchMarkets-2021}{}}%
Research \& Markets, 2021. Insights on the smart homes global market to
2026 - featuring ABB, acuity brands and emerson electric among others.
Available at:
\url{https://www.prnewswire.com/news-releases/insights-on-the-smart-homes-global-market-to-2026---featuring-abb-acuity-brands-and-emerson-electric-among-others-301425322.html}.

\end{CSLReferences}

%% chapters in the ``backmatter'' section do not have chapter numbering
%% text in the ``backmatter'' is single spaced
\backmatter
\bibliographystyle{alpha}
\bibliography{pubs}

\chapter{Appendix 1}\label{app1}

This section contains the options for the UNSW thesis class; and
layout specifications used by this thesis.

\section{Options}

The standard thesis class options provided are:

\qquad
\begin{tabular}{rl}
undergrad & default \\
hdr & \\[2ex]
11pt & default\\
12pt &\\[2ex]
oneside & default for HDR theses\\
twoside & default for undergraduate theses\\[2ex]
draft & (prints DRAFT on title page and in footer and omits pictures)\\
final & default\\[2ex]
doublespacing & default\\
singlespacing & (only for use while drafting)
\end{tabular}

\section{Margins}

The standard margins for theses in Engineering are as follows:

\qquad
\begin{tabular}{|l|r|r|}
\hline
 & U'grad & HDR\\\hline
{\verb+\oddsidemargin+} & \unit[40]{mm} & \unit[40]{mm}\\
{\verb+\evensidemargin+} & \unit[25]{mm} & \unit[20]{mm}\\
{\verb+\topmargin+} & \unit[25]{mm} & \unit[30]{mm}\\
{\verb+\headheight+} & \unit[40]{mm} & \unit[40]{mm}\\
{\verb+\headsep+} & \unit[40]{mm} & \unit[40]{mm}\\
{\verb+\footskip+} & \unit[15]{mm} & \unit[15]{mm}\\
{\verb+\botmargin+} & \unit[20]{mm} & \unit[20]{mm}\\
\hline
\end{tabular}

\section{Page Headers}

\subsection{Undergraduate Theses}
For undergraduate theses, the page header for odd numbers pages in the
body of the document is:

\quad\fbox{\parbox{.95\textwidth}{Author's Name\hfill \emph{The title of the thesis}}}

and on even pages is:

\quad\fbox{\parbox{.95\textwidth}{\emph{The title of the thesis}\hfill Author's Name}}

These headers are printed on all mainmatter and backmatter pages,
including the first page of chapters or appendices.

\subsection{Higher Degree Research Theses}
For postgraduate theses, the page header for the body of the document is:

\quad\fbox{\parbox{.95\textwidth}{\emph{The title of the chapter or appendix}}}

This header is printed on all mainmatter and backmatter pages,
except for the first page of chapters or appendices.

\section{Page Footers}

For all theses, the page footer consists of a centred page number.  
In the frontmatter, the page number is in roman numerals.  
In the mainmatter and backmatter sections, the page number is in arabic numerals.
Page numbers restart from 1 at the start of the mainmatter section.  

If the \textbf{draft} document option has been selected, then a ``Draft'' message is also inserted into the footer, as in:

\quad\fbox{\parbox{.95\textwidth}{\hfill 14\hfill\hbox to 0pt{\hss\textbf{Draft:} \today}}}

or, on even numbered pages in two-sided mode:

\quad\fbox{\parbox{.95\textwidth}{\leavevmode\hbox to 0pt{\textbf{Draft:} \today\hss}\hfill 14\hfill\mbox{}}}

\section{Double Spacing}
Double spacing (actualy 1.5 spacing) is used for the mainmatter section, except for
footnotes and the text for figures and table.

Single spacing is used in the frontmatter and backmatter sections.

If it is necessary to switch between single-spacing and double-spacing, the commands \verb+\ssp+ and \verb+\dsp+ can be used; or there is a \verb+sspacing+ environment to invoke single spacing and a \verb+spacing+ environment to invoke double spacing if double spacing is used for the document (otherwise it leaves it in single spacing).  Note that switching to single spacing should only be done within the spirit of this thesis class, otherwise it may breach UNSW thesis format guidelines.

\section{Files}

This description and sample of the UNSW Thesis \LaTeX\ class consists of a number of files:

\quad\begin{tabular}{rl}
unswthesis.cls & the thesis class file itself\\[2ex]
crest.pdf & the UNSW coat of arms, used by \verb+pdflatex+ \\
crest.eps & the UNSW coat of arms, used by \verb+latex+ + \verb+dvips+ \\[2ex]
dissertation-sheet.tex & formal information required by HDR theses\\[2ex]
pubs.bib & reference details for use in the bibliography\\[2ex]
sample-thesis.tex & the main file for the thesis
\end{tabular}

The file sample-thesis.tex is the main file for the current document (in use,
its name should be changed to something more meaningful).  It presents
the structure of the thesis, then includes a number of separate files
for the various content sections.  While including separate files is
not essential (it could all be in one file), using multiple files is
useful for organising complex work.

This sample thesis is typical of many theses; however, new authors should
consult with their supervisors and exercise judgement.

The included files used by this sample thesis are:

\quad\begin{tabular}[t]{r}
definitions.tex \\
abstract.tex \\
acknowledgements.tex \\
abbreviations.tex \\
introduction.tex \\
background.tex
\end{tabular}
\quad\begin{tabular}[t]{r}
mywork.tex \\
evaluation.tex \\
conclusion.tex \\
appendix1.tex \\
appendix2.tex 
\end{tabular}

These are typical; however the concepts and names
(and obviously content) of the files making up the matter of the
thesis will differ between theses.

\chapter{Appendix 2}\label{app2}

This section contains scads of supplimentary data.

\section{Data}

Heaps and heaps and heaps and heaps and heaps and heaps of data.
Heaps and heaps and heaps and heaps and heaps and heaps of data.
Heaps and heaps and heaps and heaps and heaps and heaps of data.
Heaps and heaps and heaps and heaps and heaps and heaps of data.
Heaps and heaps and heaps and heaps and heaps and heaps of data.

Heaps and heaps and heaps and heaps and heaps and heaps of data.
Heaps and heaps and heaps and heaps and heaps and heaps of data.
Heaps and heaps and heaps and heaps and heaps and heaps of data.
Heaps and heaps and heaps and heaps and heaps and heaps of data.
Heaps and heaps and heaps and heaps and heaps and heaps of data.

Heaps and heaps and heaps and heaps and heaps and heaps of data.
Heaps and heaps and heaps and heaps and heaps and heaps of data.
Heaps and heaps and heaps and heaps and heaps and heaps of data.
Heaps and heaps and heaps and heaps and heaps and heaps of data.
Heaps and heaps and heaps and heaps and heaps and heaps of data.

Heaps and heaps and heaps and heaps and heaps and heaps of data.
Heaps and heaps and heaps and heaps and heaps and heaps of data.
Heaps and heaps and heaps and heaps and heaps and heaps of data.
Heaps and heaps and heaps and heaps and heaps and heaps of data.
Heaps and heaps and heaps and heaps and heaps and heaps of data.

Heaps and heaps and heaps and heaps and heaps and heaps of data.
Heaps and heaps and heaps and heaps and heaps and heaps of data.
Heaps and heaps and heaps and heaps and heaps and heaps of data.
Heaps and heaps and heaps and heaps and heaps and heaps of data.
Heaps and heaps and heaps and heaps and heaps and heaps of data.

Heaps and heaps and heaps and heaps and heaps and heaps of data.
Heaps and heaps and heaps and heaps and heaps and heaps of data.
Heaps and heaps and heaps and heaps and heaps and heaps of data.
Heaps and heaps and heaps and heaps and heaps and heaps of data.
Heaps and heaps and heaps and heaps and heaps and heaps of data.

Heaps and heaps and heaps and heaps and heaps and heaps of data.
Heaps and heaps and heaps and heaps and heaps and heaps of data.
Heaps and heaps and heaps and heaps and heaps and heaps of data.
Heaps and heaps and heaps and heaps and heaps and heaps of data.
Heaps and heaps and heaps and heaps and heaps and heaps of data.

Heaps and heaps and heaps and heaps and heaps and heaps of data.
Heaps and heaps and heaps and heaps and heaps and heaps of data.
Heaps and heaps and heaps and heaps and heaps and heaps of data.
Heaps and heaps and heaps and heaps and heaps and heaps of data.
Heaps and heaps and heaps and heaps and heaps and heaps of data.

Heaps and heaps and heaps and heaps and heaps and heaps of data.
Heaps and heaps and heaps and heaps and heaps and heaps of data.
Heaps and heaps and heaps and heaps and heaps and heaps of data.
Heaps and heaps and heaps and heaps and heaps and heaps of data.
Heaps and heaps and heaps and heaps and heaps and heaps of data.

Heaps and heaps and heaps and heaps and heaps and heaps of data.
Heaps and heaps and heaps and heaps and heaps and heaps of data.
Heaps and heaps and heaps and heaps and heaps and heaps of data.
Heaps and heaps and heaps and heaps and heaps and heaps of data.
Heaps and heaps and heaps and heaps and heaps and heaps of data.

Heaps and heaps and heaps and heaps and heaps and heaps of data.
Heaps and heaps and heaps and heaps and heaps and heaps of data.
Heaps and heaps and heaps and heaps and heaps and heaps of data.
Heaps and heaps and heaps and heaps and heaps and heaps of data.
Heaps and heaps and heaps and heaps and heaps and heaps of data.



\end{document}
