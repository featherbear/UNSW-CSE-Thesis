\documentclass[]{resources/unswthesis}

%%% Class options:

%  undergrad (default)
%  hdr

%  11pt (default)
%  12pt

%  final (default)
%  draft

%  oneside (default for hdr)
%  twoside (default for undergrad)


%% Thesis details
\thesistitle{Requirements to theses submitted in the Faculty of Engineering}
\thesisschool{School of Computer Science and Engineering}
\thesisauthor{Andrew Jin-Meng Wong}
\thesisZid{z5206677}
\thesistopic{"Smart" Vacuum Cleaners - An Audit Into The Security and Integrity of IoT Systems}
\thesisdegree{Bachelor of Engineering in Computer Engineering}
\thesisdate{November 2021}
\thesissupervisor{Prof.\ Richard Buckland}% for undergrad theses only


%% My own LaTeX macros, definitions, etc
\include{definitions}


\begin{document}

%% pages in the ``frontmatter'' section have roman numeral page number
\frontmatter  
\maketitle

\include{abstract}
\include{acknowledgements}
\include{abbreviations}

\tableofcontents
\listoffigures  % if required
\listoftables  % if required

%% pages in the ``mainmatter'' section have arabic page numbers and chapters are numbered
\mainmatter

\hypertarget{introduction}{%
\chapter{Introduction}\label{introduction}}

Consumer grade Internet of Things (IoT) devices have become widely
adopted with continuously growing demand. With demands for such devices
growing by 12\% each year (Research \& Markets 2021), this AU\$130B
industry has cordially invited thousands of households to invest in
smart devices such as light bulbs, fans, televisions and fridges. Giving
the abundance and affordability of these products, IoT devices have
become an integral part of many homes, where 4 in 5 consumers would be
more inclined to choose a property over another if the former were to
have such technologies (Brown 2015).

Although convenient, these devices come with hidden costs and risks.
Behind the seemingly `simple', `smart' and `secure' product features
that attract the general populus lies a hidden complex network of
services and devices, where functionality is often obscured and private.
Without the transparency of what data is being sent, and of where that
data is being sent to, consumers inevitably pay for convenience with not
only their money but with their privacy and security (Mehic et al.
2019).

Whilst manufacturers and vendors claim to be secure and/or confidential
in how they treat \textbf{UGC} and \textbf{PII}, it lies evident from
various incidents that we cannot completely trust such claims. From
involuntary exposure of leaked Facebook user data (Abrams 2021), to
rumours of corporations monetising user data without consent (Jones
2017), there lies an equal need for consumers to understand the terms of
service to which they agree to, but additionally for companies to be
audited against those very same terms of service.

The infrastructural security and product security of IoT devices must
also be scrutinised, given the rapid product lifecycle of IoT
developments (Giese 2021). As security is often not a sellable feature
in contrast to new products and most mistakenly, convenience, proper and
wholistic security precautions are often overlooked by companies who are
more concerned with profits and high return on investments.
Consequently, the prevalence of malicious actors in the cyberworld is
alarming, where the overall lack of security awareness between consumers
invites target devices to be easily accessed with default passwords or
through unpatched vulnerabilities\footnote{https://www.shodan.io/search?query=webcam}.

The black-box nature of IoT network communication raises both privacy
and security concerns that may often be overlooked or trivialised in
exchange for convenience. This this will involve the audit of an
internet-connected robotic vacuum cleaner (Roborock S6) to assess the
internal operations and nature of data that is transmitted, as to
investigate any potential vulnerabilities that may render the device
insecure.

The outline of this thesis is as follows: In Chapter 2, we study the
motivations behind wanting to audit the privacy and security of IoT
systems. In Chapter 3, we will review existing research, results and
methods that comprise the current state of the art of security research
on robot vacuum cleaners. Gaps in existing research will then be
formalised through the thesis plan in Chapter 4. Finally we will
conclude this report in a discussion of preliminary results that will be
carried forwards into the later stages of research

\hypertarget{refs}{}
\begin{CSLReferences}{1}{0}
\leavevmode\vadjust pre{\hypertarget{ref-LawrenceAbrams-2021}{}}%
Abrams, L., 2021. 533 million facebook users' phone numbers leaked on
hacker forum. Available at:
\url{https://www.bleepingcomputer.com/news/security/533-million-facebook-users-phone-numbers-leaked-on-hacker-forum/}.

\leavevmode\vadjust pre{\hypertarget{ref-RichBrown-2015}{}}%
Brown, R., 2015. Smart homes can pay off when it's time to sell.
Available at:
\url{https://www.cnet.com/home/smart-home/what-happens-when-you-sell-your-smart-house/}.

\leavevmode\vadjust pre{\hypertarget{ref-DennisGiese-2021}{}}%
Giese, D., 2021. Smart home security \& privacy.

\leavevmode\vadjust pre{\hypertarget{ref-RhettJones-2017}{}}%
Jones, R., 2017. Roomba's next big step is selling maps of your home to
the highest bidder. Available at:
\url{https://gizmodo.com/roombas-next-big-step-is-selling-maps-of-your-home-to-t-1797187829}.

\leavevmode\vadjust pre{\hypertarget{ref-8939043}{}}%
Mehic, M., Selimovic, N. \& Komosny, D., 2019.
\href{https://doi.org/10.1109/ICAT47117.2019.8939043}{About the
connectivity of xiaomi internet-of-things smart home devices}. In
\emph{2019 XXVII international conference on information, communication
and automation technologies (ICAT)}. pp. 1--6.

\leavevmode\vadjust pre{\hypertarget{ref-ResearchMarkets-2021}{}}%
Research \& Markets, 2021. Insights on the smart homes global market to
2026 - featuring ABB, acuity brands and emerson electric among others.
Available at:
\url{https://www.prnewswire.com/news-releases/insights-on-the-smart-homes-global-market-to-2026---featuring-abb-acuity-brands-and-emerson-electric-among-others-301425322.html}.

\end{CSLReferences}

%% chapters in the ``backmatter'' section do not have chapter numbering
%% text in the ``backmatter'' is single spaced
\backmatter
\bibliographystyle{alpha}
\bibliography{pubs}

\include{appendix1}
\include{appendix2}

\end{document}
